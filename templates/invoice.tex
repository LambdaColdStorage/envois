%%%%%%%%%%%%%%%%%%%%%%%%%%%%%%%%%%%%%%%%%
% Invoice Template
% LaTeX Template
% Version 1.0 (3/11/12)
%
% This template has been downloaded from:
% http://www.LaTeXTemplates.com
%
% Original author:
% Trey Hunner (http://www.treyhunner.com/)
%
% Modified by:
% Lambda Labs
%
% License:
% CC BY-NC-SA 3.0 (http://creativecommons.org/licenses/by-nc-sa/3.0/)
%
% Important note:
% This template requires the invoice.cls file to be in the same directory as 
% the .tex file. The invoice.cls file provides the style used for structuring the
% document.
%
%%%%%%%%%%%%%%%%%%%%%%%%%%%%%%%%%%%%%%%%%

%----------------------------------------------------------------------------------------
%	DOCUMENT CONFIGURATION
%----------------------------------------------------------------------------------------
%{#

\documentclass[11pt]{article}

\usepackage[letterpaper,hmargin=0.79in,vmargin=0.79in]{geometry}
\usepackage[parfill]{parskip} % Do not indent paragraphs
\usepackage{fp} % Fixed-point arithmetic
\usepackage{calc} % Counters for totaling hours and cost
\usepackage{longtable}

\pagestyle{empty} % No page numbers
\linespread{1.5} % Line spacing

\setlength{\doublerulesep}{\arrayrulewidth} % Double rules look like one thick one

% Command for setting a default hourly rate
\newcommand{\feetype}[1]{
    \textbf{#1} 
    \\
}

% Counters for totaling up hours and dollars
\newcounter{hours} \newcounter{subhours} \newcounter{cost} \newcounter{subcost}
\setcounter{hours}{0} \setcounter{subhours}{0} \setcounter{cost}{0} \setcounter{subcost}{0}

% Formats inputed number with 2 digits after the decimal place
\newcommand*{\formatNumber}[1]{\FPround{\cost}{#1}{2}\cost} %

% Returns the total of counter
\newcommand*{\total}[1]{\FPdiv{\t}{\arabic{#1}}{1000}\formatNumber{\t}}

% Create an invoice table
\newenvironment{invoiceTable}{
    % Create a new row from title, unit quantity, unit rate, and unit name
    \newcommand*{\unitrow}[4]{%
         \addtocounter{cost}{1000 * \real{##2} * \real{##3}}%
         \addtocounter{subcost}{1000 * \real{##2} * \real{##3}}%
         ##1 & \formatNumber{##2} ##4 & \$\formatNumber{##3} & \$\FPmul{\cost}{##2}{##3}\formatNumber{\cost}%
         \\
    }
    % Create a new row from title and expense amount
    \newcommand*{\feerow}[2]{%
         \addtocounter{cost}{1000 * \real{##2}}%
         \addtocounter{subcost}{1000 * \real{##2}}%
         ##1 & & \$\formatNumber{##2} & \$\FPmul{\cost}{##2}{1}\formatNumber{\cost}%
         \\
    }

    \newcommand{\subtotalNoStar}{
        {\bf Subtotal} & {\bf \total{subhours} hours} &  & {\bf \$\total{subcost}}
        \setcounter{subcost}{0}
        \setcounter{subhours}{0}
        \\*[1.5ex]
    }
    \newcommand{\subtotalStar}{
        {\bf Subtotal} & & & {\bf \$\total{subcost}}
        \setcounter{subcost}{0}
        \\*[1.5ex]
    }
    \newcommand{\subtotal}{
         \hline
         \@ifstar
         \subtotalStar%
         \subtotalNoStar%
    }

    % Create a new row from date and hours worked (use stored fee type and hourly rate)
    \newcommand*{\hourrow}[3]{%
        \addtocounter{hours}{1000 * \real{##2}}%
        \addtocounter{subhours}{1000 * \real{##2}}%
        \unitrow{##1}{##2}{##3}{hours}%
    }
    \renewcommand{\tabcolsep}{0.8ex}
    \setlength\LTleft{0pt}
    \setlength\LTright{0pt}
    \begin{longtable}{@{\extracolsep{\fill}\hspace{\tabcolsep}} l r r r }
    \hline
    {\bf Description of Services} & \multicolumn{1}{c}{\bf Quantity} & \multicolumn{1}{c}{\bf Unit Price} & \multicolumn{1}{c}{\bf Amount} \\*
    \hline\hline
    \endhead
}{
    \hline\hline\hline
    {\bf Balance Due} & & & {\bf \$\total{cost}} \\
    \end{longtable}
}
\def \tab {\hspace*{3ex}} % Define \tab to create some horizontal white space
%#}
\begin{document}

%----------------------------------------------------------------------------------------
%	HEADING SECTION
%----------------------------------------------------------------------------------------

\hfil{\Huge\bf {{ seller.name}} }\hfil % Company providing the invoice
\bigskip\break % Whitespace
\hrule % Horizontal line

123 Broadway \hfill (000) 111-1111 \\ % Your address and contact information
City, State 12345 \hfill john@smith.com
\\ \\
{\bf Invoice To:} \\
\tab {{ buyer.name }} \\ % Invoice recipient
\tab {{ buyer.address }} \\ % Recipient's company

{\bf Date:} \\
\tab \today \\ % Invoice date

%----------------------------------------------------------------------------------------
%	TABLE OF EXPENSES
%----------------------------------------------------------------------------------------

\begin{invoiceTable}

\feetype{Consulting Services} % Fee category description

\hourrow{October 3, 2012}{8}{12} % Each separate billing day includes the date, the number of hours and the hourly rate
\hourrow{October 4, 2012}{6.5}{12}
\hourrow{October 5, 2012}{5.25}{12}
\hourrow{October 10, 2012}{9.75}{20}
\hourrow{October 11, 2012}{5}{12.51}

%------------------------------------------------

\feetype{Accounting Services} % Fee category description

\hourrow{October 10, 2012}{2}{80}
\hourrow{October 11, 2012}{1}{80}

\subtotal % Prints a subtotal, can be used multiple times

%------------------------------------------------

\feetype{Hosting Expenses} % Fee category description

\feerow{Web Hosting: October, 2012}{60} % A flat fee service, note there is no hourly rate for this

\end{invoiceTable}

%----------------------------------------------------------------------------------------

\end{document}
